In our sample, we observe a strong correlation between the $L_{max}$ and $t_2$ in $Y$ and $J$, and a weaker trend in the $H$ band. 
Using three different methods, we derive a value for $M_{^{56}Ni}$ from the $L_{max}$. We find that the derived values from the different methods are consistent with each other and within the error estimates for $M_{^{56}Ni}$. Hence, a relation between $M_{^{56}Ni}$ and $t_2$ provides direct evidence that the $t_2$ is governed by the total amount of radioactive Nickel produced in an SNIa.


Using the relation derived from the low-reddening sample, we can extrapolate an $L_{max}$ value for objects not in the low-reddening sample, but having a measured $t_2$. Therefore, this relation provides a method to deduce the bolometric peak luminosity, independent of a reddening estimate, and without requiring multi-band photometry.

On of the methods to derive $M_{^{56}Ni}$ from $L_{max}$ involves using Arnett's rule. The $\alpha$ parameter, which encodes the deviation from Arnett's, is chosen to be 1, in line with previous studies \citep[eg.][]{Stritzinger2006, Mazzali2007}. We find that the DDC models show a very small deviation from this value for the grid of input $M_{^{56}Ni}$. We find that varying the $\alpha$ according to the models doesn't change the estimated $M_{^{56}Ni}$ and hence, for our analyses, we keep $\alpha$ as 1 when using Arnett's rule.  


An example of an application of this method is the nearby SN2014J in M82, which is heavily occluded by host galaxy dust. Since this prevents 
an accurate measurement of $M_{^{56}Ni}$ from the bolometric light curves, there is a large disparity in the different values published in the literature.
Using this method, we use the relations we obtain to constrain the $M_{^{56}Ni}$. For SN2014J, we have a unique opportunity to compare different estimation methods, 
since its proximity has allowed $\gamma$ ray Co line detection and therefore, another extinction independent measurement of the $M_{^{56}Ni}$. Our value of 0.66 $\pm$ 0.15 $M_{\odot}$
compares very well with \citet{Churazov2014}, who find $M_{^{56}Ni}$ of 0.62 $\pm$ 0.13 $M_{\odot}$.   The brightness of SN2014J at late times, due to its proximity, permits us to obtain
NIR spectra at $\sim$ 300 days, which can provide an accurate measurement of the extinction and therefore, an accurate $M_{^{56}Ni}$ from the bolometric light curve. This presents
us with a confrontation of several different methods to measure the $M_{^{56}Ni}$ and hence obtain a conclusive estimate on the amount of Ni produce in this SN.

The recent discovery of $^{56}Ni$ in the outer layers of the ejecta of SN2014J \citep{Diehl2014a} offers insight into the nature of the ejecta structure. Our analysis cannot account for the Ni in the outer layers and therefore, the total amount of ${^{56}Ni}$ produced would be greater than the value of 0.66  $M_{\odot}$ we have obtained. In \citep{Diehl2014b} the authors have measured the $^{56}M_{Ni}$ from the $\gamma$ ray emission of $^{56}$Co at 847 and 1238 keV. From two different methods they obtain ranges of 0.42-0.56 ($\pm$ 0.06 $M_{\odot}$) and 0.52-0.59 ($\pm$ 0.13 $M_{\odot}$). These values are broadly consistent with our finding of 0.66 $\pm$ 0.15 $M_{\odot}$ and with the estimates summarized in Table ~\ref{tab:meth}
 
Since $\gamma$ ray line detections are unlikely for farther out SN and most of them are too faint at $\sim$ +300 days for IR spectroscopy, we apply our method to other heavily reddened SN
that are farther away than SN2014J. The first object we analyse is SN2006X. From the measurement of 0.58 $\pm$ 0.13 $M_{\odot}$, we conclude that
2006X produced the average amount of Ni for an SNIa. We compare this value to the analysis in \citet{Wang2008}, where the authors use multi-band light curves to obtain an $A_V$ value and then calculate the bolometric luminosity from which they derive the $M_{^{56}Ni}$ using Arnett's rule. Their final value of 0.5 $\pm$ 0.05 $M_{\odot}$ is consistent with the value we obtain from the NIR light curve. 
Both the test cases for SN2014J and SN2006X provide evidence for the potency of this method. 

We apply the relation for the complete sample of objects with a measured $t_2$ (excluding the already measured low-reddening sample) to derive an $L_{max}$ value and from it an $M_{^{56}Ni}$. The distribution of $M_{^{56}Ni}$ has a large variance. There is a factor of 3 difference between the lowest and highest values. We note that this sample doesn't contain faint 91bg-like objects and peculiar super-Chandra explosions, hence, the 'true' dispersion is likely to be larger.  



