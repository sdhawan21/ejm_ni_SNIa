The flux emitted by an SNIa in the UV, optical and NIR traces the comptonization of the photons emitted through the $^{56}$Ni $\rightarrow$  $^{56}$Co $\rightarrow$ $^{56}$Fe decay chain \citep[see][]{Nadyozhin1994}.
As the SN emits most of its flux in the UV to NIR passbands, the "uvoir bolometric flux" represents a physically meaningful quantity \citep{Suntzeff1996}

We select a low-reddening sample with objects that have a host extinction less than $0.1 mag$. This makes our measurements are less sensitive to a reddening law. 
For objects with sufficient amount of near maximum data in the optical and the NIR, we construct UBVRIJH bolometric light curves. We do not use $K$ band data since there are very few objects in the sample with well-sampled $K$ band light curves. Using  objects that have well-sampled $K$ light curves we calculate the flux emitted in the $K$ band and find that it is between $1-3 \%$. Thus, not using the $K$-band is not a dominant source of uncertainty. 
The magnitudes were corrected 
for reddening using a CCM reddening law for each filter. The values for the extinction are presented in table \ref{tab:mni}. The uncertainty in the reddening estimate
was propagated into the calculation of the bolometric flux
Using zero-points in the given filters, the magnitudes were converted to fluxes. The data in the different filters is interpolated, instead of using a reference filter. The filters are integrated using the trapezoidal rule.
The resulting light curve, in ergs/$cm^2$/s  was converted into an absolute bolometric light curve 
by using the distances of the SN derived from the host galaxy redshift. 

Since all distances are scaled to an $H_0=70 km s^{-1} Mpc ^{-1}$the errors in the luminosity distance are only affected by the relative errors in the 
distance moduli (see Table \ref{tab:mni} for values and uncertainty estimates). For objects not in the hubble flow, we use distance measurements from published estimates (which use others methods eg. Cepheid, Tully-Fisher relation etc.). 

In our sample, for uniformity, we restrict the analysis to objects with coverage from $u-H$ bands with coverage around the bolometric peak.

SN2001ba & 35.40 & 0.50 & $ 0.010 (0.04)$ & 0.021 (0.002) & UBVRIJH & K04a \\
SN2002dj & 31.70 & 0.30 & $ 0.020 (0.03)$ & 0.080 (0.003) & UBVRIJH & P08 \\
SN2002fk & 32.59 & 0.15 & $0.030 (0.01)$ & 0.030 (0.003) & UBVRIJH & C14 \\
SN2005M  & 35.01 & 0.09 & $0.060(0.021)$ & 0.027(0.002) & UBVRIJH & B14\\
SN2005am & 32.85 & 0.20 & $0.053(0.017)$ & 0.043(0.002) & UBVRIJH & B14\\
SN2005el & 34.04 & 0.14	& $0.015(0.012)$ & 0.098 (0.001) & UBVRIJH & B14\\
SN2005eq & 35.46 & 0.07 & $0.044(0.024)$ & 0.063(0.003) & UBVRIJH & B14\\
SN2005hc & 36.50 & 0.05 & $0.049(0.019)$ & 0.028(0.001) & UBVRIJH & B14\\
SN2005iq & 35.80 & 0.15 & $0.040(0.015)$ & 0.019(0.001) & UBVRIJH & B14\\
SN2005ki & 34.73 & 0.10 & $0.016(0.013)$ & 0.027(0.001) & UBVRIJH & B14\\
SN2006bh & 33.28 & 0.20 & $0.037(0.013)$ & 0.023(0.001) & UBVRIJH & B14\\
SN2007bd & 35.73 & 0.07 & $0.058(0.022)$ & 0.029(0.001) & UBVRIJH & B14\\
SN2007on & 31.45 & 0.08 & $<0.007$ 	& 0.010(0.001) & UBVRIJH & B14\\
SN2008R  & 33.73 & 0.16 & $0.009(0.013)$ & 0.062(0.001) & UBVRIJH & B14\\
SN2008bc & 34.16 & 0.13 & $<0.019$ 	& 0.225(0.004) & UBVRIJH & B14\\
SN2008gp & 35.79 & 0.06 & $0.098(0.022)$ & 0.104(0.005) & UBVRIJH & B14\\
SN2008hv & 33.84 & 0.15 & $0.074(0.023)$ & 0.028(0.001) & UBVRIJH & B14\\
SN2008ia & 34.96 & 0.09 & $0.066(0.016)$ & 0.195(0.005) & UBVRIJH & B14\\
SN2011fe & 28.91 & 0.20 & $0.014 (0.01)$ & 0.021(0.001) & UBVRIJH  & Pa13\\







